\documentclass{article}
\title{Newton's Law of Gravitation}
\author{ME19B067}
\begin{document}
\section{Formula}
Formula for Newton's Law of Gravitation: 
$$(F=\frac{GmM}-{r^2})$$
\section{Description}
Here, F refers to gravitational force generated due to attraction between two masses, G is the universal gravitational constant the value of $G=6.67*(10^-11)m^3kg^-1s^-2$, m & M are masses of the objects and r refers to the distance between the centres of the two masses.

The Newton's law of gravitation states that every particle in the universe attracts every other particle with a force called the gravitational force that,as shown above, is directly proportional to the product of their masses and inversely proportional to the square of the distance between the particles. The force acts along the line of intersection of both point masses. 

Though gravitational force is exerted by all masses on other masses, the force is extremely small and negligible in magnitude, especially in cases where both objects have very low masses as the value of constant G is very low. 

\end{document}

