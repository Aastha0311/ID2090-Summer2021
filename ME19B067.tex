\documentclass{article}
\title{Newton's Law of Gravitation}
\author{Aastha Mishra \\* ME19B067}
\begin{document}
\section{Formula}
Formula for Newton's Law of Gravitation: 
$$(F=\frac{GmM}-{r^2})$$
\section{Description}
Here, F refers to gravitational force generated due to attraction between two masses, G is the universal gravitational constant the value of $G=6.67*(10^-11)m^3kg^-1s^-2$, m & M are masses of the objects and r refers to the distance between the centres of the two masses. The Newton's law of gravitation states that every particle in the universe attracts every other particle with a force called the gravitational force that,as shown above, is directly proportional to the product of their masses and inversely proportional to the square of the distance between the particles. The force acts along the line of intersection of both point masses. Though gravitational force is exerted by all masses on other masses, the force often turns out to be extremely small and negligible in magnitude, especially in cases where both objects have very low masses as the value of constant G is very low.

\\ This is a very basic equation describing one of the most fundamental forces of nature which all of us have read about. It predicts how all objects around us interact and is applicable not just on Earth but on planets too. Everything around us is held together by gravitational force and it was this sheer importance of the force everywhere which got me interested in physics as a high school student and despite now having learnt about lot more complicated forces and equations, continues to draw my interest towards the subject and its applications till date.
\end{document}

